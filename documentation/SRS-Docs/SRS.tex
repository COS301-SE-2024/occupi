\documentclass[11pt,a4paper]{article}
%%%%%%%%%%%%%%%%%%%%%%%%%%%%%%%%%%%%%%%%%%%%%%%%%%%%%%%%
%                      PACKAGES                        %
%%%%%%%%%%%%%%%%%%%%%%%%%%%%%%%%%%%%%%%%%%%%%%%%%%%%%%%%

\usepackage[utf8]{inputenc}
\usepackage{graphicx} % Allows you to insert figures
\usepackage[export]{adjustbox}
\usepackage{booktabs}
\usepackage{amsmath} % Allows you to do equations
\usepackage{helvet}
\usepackage{hyperref}
\renewcommand{\familydefault}{\sfdefault}
\usepackage[a4paper, total={6.5in, 9.5in}]{geometry} % Formats the paper size, orientation, and margins
\linespread{1.1} % about 1.5 spacing in Word
\setlength{\parindent}{0pt} % no paragraph indents
\setlength{\parskip}{1em} % paragraphs separated by one line
\usepackage{listings}
\usepackage{enumitem}
\usepackage{xcolor}
\usepackage{hyperref}
\hypersetup{
	colorlinks=true,
	urlcolor=cyan,
	linktoc=none,
}
\usepackage{fancyhdr}
\pagestyle{fancy}
\fancyhead[L,C,R]{}
\fancyfoot[L]{Blix - AI Photo Editor}
\fancyfoot[C]{}
\fancyfoot[R]{\textbf{\thepage}}
\renewcommand{\headrulewidth}{0pt}
\renewcommand{\footrulewidth}{0.5pt}

\definecolor{codegreen}{rgb}{0,0.6,0}
\definecolor{codegray}{rgb}{0.5,0.5,0.5}
\definecolor{codepurple}{rgb}{0.58,0,0.82}
\definecolor{backcolour}{rgb}{0.95,0.95,0.92}

\lstdefinestyle{mystyle}{
backgroundcolor=\color{backcolour},
commentstyle=\color{codegreen},
keywordstyle=\color{magenta},
numberstyle=\tiny\color{codegray},
stringstyle=\color{codepurple},
basicstyle=\ttfamily\footnotesize,
breakatwhitespace=false,
breaklines=true,
keepspaces=true,
numbers=left,
numbersep=5pt,
showspaces=false,
showstringspaces=false,
showtabs=false,
tabsize=2,
}

\lstset{style=mystyle}
\def\code#1{\texttt{#1}}

%%%%%%%%%%%%%%%%%%%%%%%%%%%%%%%%%%%%%%%%%%%%%%%%%%%%%%%%
%            TITLE PAGE & TABLE OF CONTENTS            %
%%%%%%%%%%%%%%%%%%%%%%%%%%%%%%%%%%%%%%%%%%%%%%%%%%%%%%%%

\begin{document}

\begin{titlepage}
    \centering
    % \includegraphics[width=0.5\textwidth]{your_logo.png}\par\vspace{1cm}
    {\scshape\LARGE Software Requirements Specification\par}
    \vspace{1.5cm}
    {\huge\bfseries Occupi - Office capacity predictor\par}
    \begin{figure}[h]
        \centering % center the image
        \includegraphics[width=0.5\textwidth]{../pics/blix.png}
    \end{figure}
    \vspace{2.5cm}
    {\Large\itshape The Spanish Inquisition\par}
    \begin{tabular}{|c|c|}
        \hline
        \textbf{Name}      & \textbf{Student Number} \\
        \hline
        Rethakgetse Manaka & u22491032               \\
        Kamogelo Moeketse  & u22623478               \\
        Michael Chinyama   & u21546551               \\
        Tinashe            & u--------               \\
        Carey              & u---------              \\
        \hline
    \end{tabular}
    \vfill
    {\large \today\par}
\end{titlepage}

\tableofcontents
\pagebreak

%%%%%%%%%%%%%%%%%%%%%%%%%%%%%%%%%%%%%%%%%%%%%%%%%%%%%%%%
%                MAIN DOCUMENT CONTENT                 %
%%%%%%%%%%%%%%%%%%%%%%%%%%%%%%%%%%%%%%%%%%%%%%%%%%%%%%%%

\addcontentsline{toc}{section}{Introduction}
\section*{Introduction}

\addcontentsline{toc}{subsection}{Purpose and Vision}
\subsection*{Purpose and Vision}


\pagebreak


\addcontentsline{toc}{section}{Specifications}
\section*{Specifications}

\addcontentsline{toc}{subsection}{Hardware Requirements}
\subsection*{Hardware Requirements}
\begin{itemize}


\end{itemize}

\addcontentsline{toc}{subsection}{System Requirements}
\subsection*{System Requirements}

Occupi is compatible with the following operating systems:
\begin{itemize}
    \item[\textbullet] \textbf{Windows}
    \item[\textbullet] \textbf{Linux }.
    \item[\textbullet] \textbf{MacOS }.
\end{itemize}


\addcontentsline{toc}{section}{User Stories and Characteristics}
\section*{User Stories and Characteristics}

\addcontentsline{toc}{subsection}{Characteristics}
\subsection*{Characteristics}

\subsubsection*{Novice Users}


\subsubsection*{Intermediate Users}


\subsubsection*{Professional Users}


\pagebreak

\addcontentsline{toc}{subsection}{Stories}
\subsection*{Stories}

\subsubsection*{Novice users}


\subsubsection*{Intermediate users}




\subsubsection*{Professional users}


\pagebreak

\addcontentsline{toc}{section}{Class Diagrams}
\section*{Class Diagrams}


\clearpage


\pagebreak

\addcontentsline{toc}{section}{Functional Requirements}
\section*{Functional Requirements}

\addcontentsline{toc}{subsection}{Requirements}
%I split these up but idk, probs wrong sus
\subsection*{Requirements}

Requirements with a \textbf{*} next to it represents optional and \textit{nice
    to have} requirements.

\begin{enumerate}[label=\arabic*.]
    \item Project Management
          \begin{enumerate}[label*=\arabic*.]
              \item Users should be able to import a single photo as well as multiple photos.
              \item The system should be able to export the unedited images and the edited images.
              \item The system should store projects and retrieve projects from local storage.
              \item The system should be able to sync projects to the cloud. \textbf{*}
              \item The system should allow the user to revert changes.
              \item Users should be able to efficiently switch between and work on multiple projects.
              \item Users should be able to share their projects, images, and graphs.
          \end{enumerate}

    \item Layout
          \begin{enumerate}[label*=\arabic*.]
              \item Users should be able to view the changes they made to the original image,
                    either with a button or slider.
              \item With the use of image segmentation the users should be able select
                    objects in the image by pressing on the object.
              \item Users should be able to customize the layout tiles to fit their preference.
              \item System should be able to display multiple graphs and photos
              \item System should have a command palette to to provide easy access to tools.
          \end{enumerate}

    \item Graph Management
          \begin{enumerate}[label*=\arabic*.]
              \item Add nodes
              \item Remove nodes
              \item Manipulate node positions
              \item Anchor nodes to each other to create a logical data flow
              \item Change the properties and values of nodes
              \item System should contain a collision detection algorithm to
                    place the nodes on the graph
              \item Cycle detection algorithm to indicate invalid connections
          \end{enumerate}

    \item Plugin Management
          \begin{enumerate}[label*=\arabic*.]
              \item Users should be able to create their own plugins
              \item Users should be able to load plugins of their choice
              \item Users should be able to load plugins in real time
              \item Plugin functionality should be available in real time once the plugin is loaded
          \end{enumerate}

    \item Natural Language Processing to Graph \textbf{*}
          \begin{enumerate}[label*=\arabic*.]
              \item Generated graph should conform to a specific grammar
              \item User should be able to describe an edit the image.
                    which should be processed to generate a functional graph.
          \end{enumerate}

    \item Photo Processing and Graph Interpreter
          \begin{enumerate}[label*=\arabic*.]
              \item The system should be able to support multiple different image formats/file
                    types, such as jpeg, png, svg etc
              \item Users should be able to edit the photos manually with the standard
                    features such as adjusting the white balance, hue, and exposure etc.
              \item Users should be able to edit a selected object in the image the
                    same way one would edit the rest of the image.
              \item System should be able to do object segmentation.
          \end{enumerate}

    \item User Administration
          \begin{enumerate}[label*=\arabic*.]
              \item Register and sign in with multiple providers
              \item View and manage account details
              \item View and manage list of projects
              \item Manage syncing of projects, layouts and user settings across multiple
                    devices\textbf{*}
          \end{enumerate}
\end{enumerate}

\addcontentsline{toc}{subsection}{Subsystems}
\subsection*{Subsystems}
\begin{enumerate}
    \item Project Management
    \item Layout
    \item Graph Management
    \item Plugin Management
    \item Natural Language Processing to Graph
    \item Photo Processing \& Graph Interpreter
    \item User Administration
\end{enumerate}

\pagebreak

\addcontentsline{toc}{subsection}{Use Case Diagrams}
\subsection*{Use Case Diagrams}

\subsubsection*{Project Management Subsystem}
\begin{figure}[htbp]
    \centering
    \includegraphics[width=1\textwidth]{../diagramPng/Usecase Project-Subsystem.png}
\end{figure}

\pagebreak
\subsubsection*{Layout Subsystem}
\begin{figure}[htbp]
    \centering
    \includegraphics[width=1\textwidth]{../diagramPng/Usecase Layout-Subsystem.png}
\end{figure}

\subsubsection*{Graph Management Subsystem}
\begin{figure}[htbp]
    \centering
    \includegraphics[width=0.6\textwidth]{../diagramPng/Usecase Graph-Subsystem.png}
\end{figure}

\pagebreak
\subsubsection*{Image Processing Subsystem}
\begin{figure}[htbp]
    \centering
    \includegraphics[width=0.6\textwidth]{../diagramPng/UseCase Image-Subsystem.png}
\end{figure}

\subsubsection*{User Management Subsystem}
\begin{figure}[htbp]
    \centering
    \includegraphics[width=0.6\textwidth]{../diagramPng/Usecase User-Subsystem.png}
\end{figure}


\pagebreak

\addcontentsline{toc}{section}{Quality Requirements}
\section*{Quality Requirements}

\subsection*{Performance}

Performance is pivotal to the success of this project. A bad performing system
will severly hamper the user experience and will make the systems survival unfeasible.
Due to the nature of photo editing, the system should be able to handle large
images and projects without significant lag or delay, thus it is vital that these tasks are
not just performed well, but exceedingly well to ensure a good user experience.

    {\bf Quantification}

Performance is quantified by the maximum processing time for standard photo editing operations.
The Throughput and resource utilization of the system must also be investigated for the purposes
of the Blix system.

The maximum processing time of the system is measured by the time it takes to perform a standard photo editing operation
immediately after the user has requested the operation. The standard photo editing operations are defined as the following:
\begin{enumerate}
    \item Adjusting the white balance
    \item Adjusting the hue
    \item Adjusting the exposure
    \item Adjusting the saturation
    \item Rotating the image
    \item Cropping the image
    \item Applying a filter to the image
\end{enumerate}

The throughput of the system is measured by the number of standard photo editing operations that can be performed in a given time period.

The resource utilization of the system is measured by the amount of memory and CPU usage of the system during runtime.


    {\bf Targets}

The target maximum processing time of the system is 1 second for a standard photo editing operations

The target throughput of the system is 5 standard photo editing operations per second.

The target resource
utilization of the system is :
\begin{enumerate}
    \item Less than 100 MB of memory and 10\% CPU usage for the minor photo editing operations
    \item 500MB of memory and 50\% CPU usage for the standard photo editing operations.
    \item 1GB of memory and less than 90\% CPU usage for the major photo editing operations.
\end{enumerate}

\subsection*{Reliability}

The reliability of the system is dependent on the ability of the system to prevent and recover from failures.
The system should be able to prevent failures by ensuring that the system is always in a consistent state.
The system should be able to recover from failures by ensuring that the system can be restored to a consistent state, such that
the user does not lose any work.

    {\bf Quantification}

The reliability of the system is quantified by the the mean time between failures (MTBF) and the mean time to recovery (MTTR).
Another metric that will be used to quantify the reliability of the system is the number of critical failures per month.

The mean time between failures is measured by the number of operations performed by the system before a failure occurs. This number
is then divided by the number of failures that occured during the time period.


The mean time to recovery is measured by the number of operations required to restore the system to a consistent state after a failure has occured.

The number of critical failures per month is measured by the number of failures that occured during the month that resulted in the loss of data or the loss of the ability to perform photo editing operations.

    {\bf Targets}

The target mean time between failures is 100 operations.

The target mean time to recovery is 10 operations.

The target number of critical failures per month is 0.



\subsection*{Usability}

The usability of the system is dependent on the ability of the system to be easily used by the user. It is important that the system is easy to use and that the
user is able to perform the desired tasks with ease and with clearly defined steps. The system caters for all users, from novice to expert, thus it is important to note that not all users
will be familiar with the system and features, thus the system must provide alternatives for these users

    {\bf Quantification}

To properly quantify Usability, user satisfaction cannot be neglected. Additionally the learning curve of the system must be investigated to ensure that the system is easy to use.
Finally the number of user requests completed per month must be investigated to ensure that the system is able to appeal to the users.

User satisfaction is measured by the number of users that are satisfied with the system. This number is then divided by the total number of users that used the system during the time period.

The learning curve of the system is measured by the number of operations required to perform a standard photo editing operation. This number is then divided by the number of operations required to perform the same operation on a standard photo editing software.

The number of user requests completed per month is measured by the number of requests that were fulfilled during the month. This number is then divided by the total number of requests that were made by users during the month.

    {\bf Targets}

The target user satisfaction is 90\%.

The target learning curve is 1.5 times the number of operations required to perform the same operation on a standard photo editing software.

The target number of user requests completed per month is 60\%.

\subsection*{Security}

Security is an extension of the reliability of the system. It is important that the system is able to prevent and recover from security breaches to protect user.
At the same time, extensive security hampers the usability of the system, thus it is important to find a balance between security and usability to provide the best

    {\bf Quantification}

Security will be conceptually quantified by the integrity , confidentiality and availability of the system.

The integrity of the system is measured by the systems transparency regarding the users limitations the policies of the system.

The confidentiality of the system is measured by how secure the user's personal details are from other users. This is measured by the amount of security measures that are in place and the number of violations that have occured.

The availability of the system is measured by the amount of systems that the users are provided access to to customize and modify.

    {\bf Goals}

The integrity of the system must be well defined and transparent to the user.

The confidentiality of the system must be extremely well guarded and the number of violations must be 0.

The system must have a high availability such that the users must be able to customize and modify the system to their liking.

\subsection*{Compatibility}

Compatibility is an extremely important quality requirement for the system. The system must be able to run on a variety of platforms and
must be able to support a variety of image formats. Due to the nature of the system, extensive compatibility is required to ensure that the system is able to
appeal to a wide range of users.

    {\bf Quantification}

Compatibility is quantified by the number of platforms that the system is able to run on and the number of image formats that the system is able to support.

The number of platforms that the system is able to run on is measured by the number of platforms that the system is able to run on, with a specific focus on the most popular platforms.

The number of image formats that the system is able to support is measured by the number of image formats that the system is able to support, with a specific focus on the most popular image formats.

    {\bf Goals}

The system should be compatible with all the most well-known platforms :
\begin{enumerate}
    \item Windows
    \item Linux
    \item Mac OS
\end{enumerate}

The system should be compatible with all the most well-known image file formats :
\begin{enumerate}
    \item JPEG
    \item TIFF
    \item PNG
    \item GIF
    \item BMP
\end{enumerate}

\end{document}